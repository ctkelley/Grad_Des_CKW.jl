% SIAM Shared Information Template
% This is information that is shared between the main document and any
% supplement. If no supplement is required, then this information can
% be included directly in the main document.


% Packages and macros go here
\usepackage{lipsum}
\usepackage{amsfonts}
\usepackage{amssymb}
\usepackage{mathrsfs}
\usepackage{graphicx}
\usepackage{epstopdf}
\usepackage{enumerate}
\usepackage{algorithmic}
\usepackage{xurl}
\ifpdf
  \DeclareGraphicsExtensions{.eps,.pdf,.png,.jpg}
\else
  \DeclareGraphicsExtensions{.eps}
\fi

\usepackage{subfigure}

\usepackage[ruled,vlined,algo2e]{algorithm2e}
%linesnumbered,inoutnumbered,
\crefname{algorithm2e}{Algorithm}{Algorithms}
\crefname{algocfline}{Algorithm}{Algorithms}
\crefname{algocf}{Algorithm}{Algorithms}

\usepackage{listings,matlab-prettifier}
\lstset{style=Matlab-editor}
\lstset{language=Matlab}

\newcommand{\bR}{\mathbb{R}}
\newcommand{\bN}{\mathbb{N}}

\newcommand{\cF}{\mathcal{F}}
\newcommand{\cH}{\mathcal{H}}

\newcommand{\rmd}{\mathrm{d}}

\newcommand{\bfb}{\mathbf{b}}
\newcommand{\bfc}{\mathbf{c}}
\newcommand{\bfe}{\mathbf{e}}
\newcommand{\bfu}{\mathbf{u}}
\newcommand{\bfv}{\mathbf{v}}
\newcommand{\bfw}{\mathbf{w}}
\newcommand{\bfx}{\mathbf{x}}
\newcommand{\bfy}{\mathbf{y}}
\newcommand{\bfz}{\mathbf{z}}

\newcommand{\bfU}{\mathbf{U}}

\newcommand{\bfA}{\mathbf{A}}
\newcommand{\bfF}{\mathbf{F}}
\newcommand{\bfL}{\mathbf{L}}

\newcommand{\Rn}{\mathbb{R}^{n}}
\newcommand{\Rnn}{\mathbb{R}^{n \times n}}

\newcommand{\zz}{^{\top}}

\newcommand{\uast}{^{\ast}}

\newcommand{\proj}{\mathsf{\Pi}}
\newcommand{\sign}{\mathsf{sign}}
\newcommand{\Diag}{\mathsf{Diag}}

\newcommand{\dkh}[1]{\left(#1\right)}
\newcommand{\hkh}[1]{\left\{#1\right\}}
\newcommand{\fkh}[1]{\left[#1\right]}
\newcommand{\jkh}[1]{\left\langle#1\right\rangle}
\newcommand{\norm}[1]{\left\|#1\right\|}
\newcommand{\abs}[1]{\left\lvert #1\right\rvert}

\newcommand{\comm}[1]{{\color{red}#1}}
\newcommand{\revise}[1]{{\color{blue}#1}}
\newcommand{\remove}[1]{{\color{Gray}\sout{#1}}}

% Add a serial/Oxford comma by default.
\newcommand{\creflastconjunction}{, and~}

% Used for creating new theorem and remark environments
\newsiamremark{remark}{Remark}
\newsiamremark{hypothesis}{Hypothesis}
\crefname{hypothesis}{Hypothesis}{Hypotheses}
\newsiamthm{claim}{Claim}
\newsiamremark{fact}{Fact}
\crefname{fact}{Fact}{Facts}
\newsiamremark{example}{Example}
\crefname{example}{Example}{Examples}
\newsiamthm{assumption}{Assumption}
\crefname{assumption}{Assumption}{Assumptions}

% Sets running headers as well as PDF title and authors
\headers{Complexity with H{\"o}lder Continuous Gradient Terms}{X. Chen, C. T. Kelley, and L. Wang}

% Title. If the supplement option is on, then "Supplementary Material"
% is automatically inserted before the title.
\title{Complexity of Projected Gradient Methods for Strongly Convex Optimization with H{\"o}lder Continuous Gradient Terms\thanks{Submitted to the editors 22 January, 2026.
\funding{We would like to acknowledge support for this project from RGC grant JLFS/P-501/24 for the CAS AMSS-PolyU Joint Laboratory in Applied Mathematics and Hong Kong Research Grant Council project PolyU15300024.}}}

%\subtitle{}

% Authors: full names plus addresses.
\author{
	Xiaojun Chen\thanks{Department of Applied Mathematics, The Hong Kong Polytechnic University, Hong Kong, China (\email{maxjchen@polyu.edu.hk}).}
	\and C. T. Kelley\thanks{Department of Mathematics, Box 8205, North Carolina State University, Raleigh, NC 27695-8205, USA (\email{Tim\_Kelley@ncsu.edu}).}
	\and Lei Wang\thanks{Department of Applied Mathematics, The Hong Kong Polytechnic University, Hong Kong, China (\email{lei2wang@polyu.edu.hk}).}
}

\usepackage{amsopn}
\DeclareMathOperator{\diag}{diag}
\DeclareMathOperator*{\argmin}{arg\,min}
\DeclareMathOperator*{\argmax}{arg\,max}

%%% Local Variables:
%%% mode:latex
%%% TeX-master: "ex_article"
%%% End:
