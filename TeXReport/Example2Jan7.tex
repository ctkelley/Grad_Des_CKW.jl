\subsection{Example 2}
\label{subsec:example2}

We consider a second
numerical example motivated by a semi-linear elliptic problem with a constraint on the solution in a certain set \cite{Tang2025uniqueness}.
Let $D=(0,1)^2$ and
\begin{equation} \label{eq:cF2}
        \cH (u) = - \Delta u + \lambda |u|^{\nu} - |u|^{p} u
\end{equation}
on  $D$ with the boundary condition $u=1$ on the boundary $\partial D$, 
where $p > 1$, $\nu \in (0, 1)$ and $\lambda > p/\nu$ are  constants.
We consider the variational inequality that is to find $u^*\in [-1,1]$ such that
for any $u\in [-1,1]$,
\begin{equation*}
        \cH(u^*)(u-u^*)\ge 0.
\end{equation*}
This problem is equivalent to the nonlinear equation
\begin{equation}\label{exampleVI}
        0=\cF(u):=\left\{\begin{array}{ll}
                \cH(u)  & \quad {\rm if} \quad u-\cH(u) \in [-1, 1],\\
                u-1     & \quad {\rm if}  \quad u-\cH(u) \ge 1,\\
                u+1     & \quad {\rm otherwise.}
        \end{array}\right.
\end{equation}
Discretizing \cref{eq:cF2} with the standard five point difference scheme \cite{LeVeque2007finite}, problem~\cref{exampleVI} leads to the following  system of nonlinear equations
\begin{equation}\label{example2}
        \bfF(\bfu) = \bfu-{\mathrm \Pi}_{\bfU}\Big(\bfu- \tau(\bfA \bfu + \lambda |\bfu|^{\nu} - |\bfu|^{p -1}\bfu - \bfb)\Big) = 0,
\end{equation}
where $\bfU=[-1,1]^n$,  $\tau>0$ is a constant, $ \bfA\in \mathbb{R}^{n\times n}$ is a symmetric positive definite matrix and $\bfb\in \mathbb{R}^n.$  Note that \cref{example2} is the first-order optimal condition of the minimization problem
\begin{equation}\label{example2min}
        \min_{\bfu \in [-1,1]^n} f(\bfu):= \frac{1}{2}\bfu^\top\bfA\bfu + \frac{\lambda}{1+\nu} \bfe\zz |\bfu|^{\nu + 1}- \frac{1}{1+p}\bfe\zz  \max(\bfu, -\bfu)^{p+1} + \bfb^\top \bfu.
\end{equation}
The Hessian matrix of $f$ at $\bfu$ with $\bfu_i\neq 0$, $i=1,\ldots,n$ has the form
$$\nabla^2 f(\bfu)=\bfA  + \lambda \nu |\bfu|^{\nu-1} -p {\rm diag} \Big(\max (-\bfu, \bfu)^{p-1}\Big),$$
Since $\lambda\nu>p$, $\nabla^2 f(\bfu)$ is
symmetric positive definite for any $\bfu\in [-1, 1]^n$ with $\bfu_i\neq 0$, $i=1,\ldots,n$. Hence $f$ is $\mu$-strongly convex in $[-1,1]^n$ with $\mu=\lambda_{\min}(\bfA)$  and the system \cref{example2} has a unique solution in
$[-1, 1]^n.$ However, $\nabla f$ is not  Lipschitz continuous in $[-1,1]^n.$

Let
$$f_1(\bfu)=\frac{1}{2}\bfu^\top\bfA\bfu + \bfb^\top \bfu,  f_2(\bfu)=\frac{\lambda}{1+\nu} \bfe\zz |\bfu|^{\nu + 1}, f_3(\bfu)=- \frac{1}{1+p}\bfe\zz  \max(\bfu, -\bfu)^{p+1}$$
This example satisfies \cref{asp:function} (ii) with $L_1=\lambda_{\max}(\bfA)$, $L_2=\lambda\nu$, $L_3=pn^{\frac{1}{2}}, \alpha_1=\alpha_3=1, \alpha_2={1-\nu}$.

\subsubsection{Results}
\label{subsubsec:results2}
In this example we do not have an analytic solution, so we only
plot the residual norms $\| \cF (\bfu) \|$.

Clearly the only interesting cases for this example are ones where the
solution can be negative. One such case, which we use here, is
\[
\nu=p=.1, \lambda=40.
\]
As was the case for Example 1, problems where the exponent for the 
non-Lipschitz term is small are difficult. In particular one cannot 
drive the residual to a small value. We compare Algorithm 1 and 
Algorithm 3. We use stepsizes of $.1 h^2$ for Algorithm 1 and
$20 h^2$ for Algorithm 3.
Figure~\ref{fig:ex2} shows that Algorithm 3 benefits from the larger
step size, but that we can only obtain a modest reduction in the
residual norm in both cases.

\begin{figure}[h!]
        \centering
        \subfigure[Algorithm 1]{
                \label{subfig:alg1e2}
                \includegraphics[width=0.45\linewidth]{Figures/alg1e2.pdf}
        }
        \subfigure[Algorithm 3]{
                \label{subfig:alg3e2}
                \includegraphics[width=0.45\linewidth]{Figures/alg3e2.pdf}
        }
        \caption{Numerical performance of Algorithm 1 for problem~\cref{opt:test}.}
        \label{fig:ex2}
\end{figure}

