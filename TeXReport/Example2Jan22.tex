\subsection{Semi-linear elliptic problem with a constraint}
\label{subsec:example2}


\begin{figure}[ht]
	\centering
	\subfigure[\cref{alg:gd}]{
		\label{subfig:alg1e2t1}
		\includegraphics[width=0.45\linewidth]{Figures/alg1e2t1.pdf}
	}
	\subfigure[\cref{alg:ufgm}]{
		\label{subfig:alg3e2t1}
		\includegraphics[width=0.45\linewidth]{Figures/alg3e2t1.pdf}
	}
	
	\subfigure[\cref{alg:gd}]{
		\label{subfig:alg1e2t2}
		\includegraphics[width=0.45\linewidth]{Figures/alg1e2t2.pdf}
	}
	\subfigure[\cref{alg:ufgm}]{
		\label{subfig:alg3e2t2}
		\includegraphics[width=0.45\linewidth]{Figures/alg3e2t2.pdf}
	}
	\caption{Numerical performance of \cref{alg:gd} and \cref{alg:ufgm} for problem~\cref{opt:test2} with different values of $\alpha$.}
	\label{fig:ex2}
\end{figure}

We consider a second numerical example motivated by a semi-linear elliptic problem with a constraint on the solution in a certain set \cite{Tang2025uniqueness}. 
Let
\begin{equation}
	\label{eq:cF2}
	\cH (u) = - \Delta u + \delta |u|^{\alpha} \sign (u) - |u|^{p - 1} u,
\end{equation}
on $D = (0, 1)^2$ with the boundary condition $u (x, y) = 0.5 - \sin(x) \sin(y)$ on $\partial D$. 
Here, $\alpha \in (0, 1)$, $p > 1$, and $\delta > p / \alpha$ are three constants. 
We consider the variational inequality that is to find $u\uast \in [-1,1]$ such that
\begin{equation*}
	\cH (u\uast) (u - u\uast) \geq 0,
\end{equation*}
for any $u\in [-1,1]$. 
This problem is equivalent to the following nonlinear equation,
\begin{equation}
	\label{exampleVI}
    0 = \cF (u) :=
    \left\{
    \begin{aligned}
        & \cH (u), 
        && \mbox{if~} u - \cH (u) \in [-1, 1], \\
        & u - 1, 
        && \mbox{if~} u - \cH(u) \geq 1, \\
        & u + 1,
        && \mbox{otherwise}.
    \end{aligned}
    \right.
\end{equation}
By discretizing \cref{eq:cF2} with the standard five point difference scheme \cite{LeVeque2007finite}, problem~\cref{exampleVI} leads to the following  system of nonlinear equations,
\begin{equation}
	\label{example2}
	0 = \bfF (\bfu) := \bfu - \proj_{\bfU} \dkh{ \bfu - \tau \dkh{ \bfA \bfu + \delta \abs{\bfu}^{\alpha} \sign (\bfu) - \abs{\bfu}^{p - 1} \bfu - \bfb } },
\end{equation}
where $\bfU = [-1, 1]^n$,  $\tau > 0$ is a constant, $\bfA \in \Rnn$ is a symmetric positive definite matrix, and $\bfb \in \Rn$ encodes the boundary conditions. 
Note that \cref{example2} is the optimality condition of the following problem,
\begin{equation}
	\label{opt:test2}
	\min_{\bfu \in \bfU} \hspace{2mm}
	f (\bfu):= \frac{1}{2}\bfu\zz \bfA \bfu + \frac{\delta}{1 + \alpha} \bfe\zz \abs{\bfu}^{1 + \alpha} - \frac{1}{1 + p} \bfe\zz \abs{\bfu}^{1 + p} - \bfb\zz \bfu.
\end{equation}
The Hessian matrix of $f$ at $\bfu$ with $\bfu_i \neq 0 \; (i = 1, \dotsc, n)$ has the form
\begin{equation*}
	\nabla^2 f (\bfu) = \bfA + \delta \alpha \Diag \dkh{\abs{\bfu}^{\alpha - 1}} - p \Diag \dkh{ \abs{\bfu}^{p - 1}},
\end{equation*}
Since $\delta > p / \alpha$, $\nabla^2 f (\bfu)$ is
symmetric positive definite for any $\bfu \in \bfU$ with $\bfu_i \neq 0 \; (i = 1, \dotsc, n)$. Hence, the function $f$ is $\mu$-strongly convex in $\bfU$ with $\mu = \lambda (\bfA)$ and the system \cref{example2} has a unique solution in $\bfU$. 
%However, $\nabla f$ is not Lipschitz continuous in $\bfU$.
The optimization model \eqref{opt:test2} is a special instance of problem~\eqref{opt:main} with $\Omega = \bfU$, $m = 2$,
\begin{equation*}
	f_1 (\bfu) = \bfu\zz \bfA \bfu - 2 \bfb\zz \bfu - \frac{2}{1 + p} \bfe\zz \abs{\bfu}^{1 + p}, 
	\mbox{~and~}
	f_2 (\bfu) = \frac{2 \delta}{1 + \alpha} \bfe\zz \abs{\bfu}^{1 + \alpha}.
\end{equation*}
It is clear that Assumption~\ref{asp:function} (ii) holds with $\alpha_1 = 1$, $L_1 = 2 \norm{\bfA} + 2 p$, $\alpha_2 = \alpha$, and $L_2 = 2 \delta \alpha$.

%\subsubsection{Results}
%\label{subsubsec:results2}
In this example, we do not have an analytic solution and we only plot the residual norm $\norm{ \bfF (\bfu) }$ with $\tau$ being the stepsize. 
We compare the performance of \cref{alg:gd} and \cref{alg:ufgm} on problem~\eqref{opt:test2} with $p = 1.5$ and $\delta = 20$. 
The stepsizes of \cref{alg:gd} and \cref{alg:ufgm} are set to $\tau = 0.1 h^2$ and $\tau = 20 h^2$, respectively. 
%\Cref{fig:ex2t1} and \Cref{fig:ex2t2} illustrate the performance of two algorithms for $\alpha \in \{0.1, 0.2, 0.3, 0.4\}$ and $\alpha \in \{0.5, 0.6, 0.7, 0.8\}$, respectively. 
%\Cref{fig:ex2} illustrates the performance of two algorithms for different values of $\alpha$. 
\Cref{subfig:alg1e2t1} and \Cref{subfig:alg1e2t2} present the performance of \Cref{alg:gd} for $\alpha \in \{0.1, 0.2, 0.3, 0.4\}$ and $\alpha \in \{0.5, 0.6, 0.7, 0.8\}$, respectively. 
In a similar vein, \Cref{subfig:alg3e2t1} and \Cref{subfig:alg3e2t2} illustrate the behavior of \Cref{alg:ufgm} across the same ranges of $\alpha$. 
Similar as the case in \cref{subsec:example}, problems where the exponent $\alpha$ for the non-Lipschitz term in the gradients is small are difficult. 
In particular, one cannot drive the residual to a small value. 
For larger values of $\alpha$, both algorithms demonstrate strong performance. 
Furthermore, \cref{alg:ufgm} exhibits a faster convergence rate, benefiting from the use of a larger stepsize. 





%Clearly, the only interesting cases for this example are ones where the
%solution can be negative. One such case, which we use here, is
%\begin{equation*}
%	\alpha = p = 0.1, \delta = 40.
%\end{equation*}
%As was the case for \cref{subsec:example}, problems where the exponent for the non-Lipschitz term in the gradients is small are difficult. 
%In particular one cannot drive the residual to a small value. 
%We compare \cref{alg:gd} and \cref{alg:ufgm}. 
%We use stepsizes of $0.1 h^2$ for \cref{alg:gd} and
%$20 h^2$ for \cref{alg:ufgm}. 
%\Cref{fig:ex2} shows that \cref{alg:ufgm} benefits from the larger step size, but that we can only obtain a modest reduction in the residual norm in both cases. 



%\begin{figure}[h!]
%	\centering
%	\subfigure[\cref{alg:gd}]{
%		\label{subfig:alg1e2t2}
%		\includegraphics[width=0.45\linewidth]{Figures/alg1e2t2.pdf}
%	}
%	\subfigure[\cref{alg:ufgm}]{
%		\label{subfig:alg3e2t2}
%		\includegraphics[width=0.45\linewidth]{Figures/alg3e2t2.pdf}
%	}
%	\caption{Numerical performance of \cref{alg:gd} and \cref{alg:ufgm} for problem~\cref{opt:test2} with larger values of $\alpha$.}
%	\label{fig:ex2t2}
%\end{figure}

