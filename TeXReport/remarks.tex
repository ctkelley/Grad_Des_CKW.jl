\documentclass{article}
\usepackage{tim,epsfig,fullpage,algorithm,algorithmic,ma580}
\usepackage{url}
\usepackage{graphicx} 
\usepackage{doi}
\usepackage{cite}
\usepackage{hyperref}


\begin{document}

Here are some remarks on Example 4.4 in \cite{QuBianChen}. I have
corrected the formulae in the example and support my results with a
grid refinement study. 

I am planning to test some solvers and use initial iterates that
satisfy the boundary conditions next. An initial iterate that does
not satisfy the boundary conditions is not a good choice for a PDE
problem. It is easy to compute one that does by solving $\nabla^2 w = 0$
with the boundary conditions.

The example is  on $\Lambda = (0,1) \times (0,1)$
\begeq
\label{eq:bvp}
-\nabla^2 u + N(u) = c
\endeq
subject to boundary conditions
\begeq
\label{eq:bc}
u = v \mbox{ on } \partial \Lambda
\endeq
and a free boundary condition
\begeq
\label{eq:freebc}
u = 0 \mbox{ in } \Lambda_0 \mbox{ and }
u = \| \nabla u \| = 0 \mbox{ on } \partial \Lambda_0.
\endeq
The domain $\Lambda_0$ is unknown. As is stands this problem is not well
posed unless you allow $\Lambda_0$ to be empty because for a general $N$
and $c$ there is no guarantee that $u=0$ anywhere in $\Lambda$.

In this particular case
\[
N(u) = \frac{9}{(1 - p)^2} u^p + \delta e^{-u}.
\]
and you generate $c$ so the solution of the problem is
\[
v(\xi, \zeta) = \left(\frac{3 r - 1}{2} \right)^{2p/(1-p)} \max(0, r-1/3)
\]
So $\Lambda_0 = \{ (\xi,\zeta) , | , \sqrt{\xi^2 + \zeta^2} > 1/3 \}$

Once you have done this, the free boundary has been encoded in the right
side $c$ and the boundary conditions, so simply solving the equation will
recover the free boundary with no additional work.

So, we must have
\begeq
\label{eq:cform}
c = -\nabla^2 v + N(v).
\endeq
I will show that 
\[
\nabla^2 v = (3/2) \sigma (\sigma + 1 ) T^{\sigma-1} 
+ \frac{1}{r} (\sigma+1) T^\sigma 
\]
Here 
\[
T = \frac{3 r - 1}{2} \max(0, r-1/3) \chi^*(r)
\mbox{ and } \sigma = 2p/(1-p)
\]
where $\chi^*$ is the characteristic function of $(1/3,\infty)$.
So, we have
\[
v = T^\sigma \max(0, r-1/3) = \frac{2}{3} T^{\sigma + 1}.
\]

\section{Laplacian in Polar Coordinates}

We will use the fact that $v$ depends only of $r$ and not the angular variable
to obtain
\[
\nabla^2 v = v_{rr} + \frac{1}{r} v_r
\]
For $r > 1/3$, $T_r = 3/2$ so
\[
v_r = (\sigma + 1) T^\sigma
\]
and 
\[
v_{rr} = \sigma (\sigma+1) T^{\sigma-1} (3/2).
\]

So my claim is that 
\begeq
\label{eq:rlap}
\nabla^2 v = (\sigma + 1) (T^\sigma + (3/2) \sigma T^{\sigma-1}).
\endeq
This is a calculation that is easy to compute incorrectly because one must
get the boundary conditions incorporated in to the discrete equations in
the proper order. You do this in \cite{QuBianChen} with the vector
$\tilde v$ which you add to $c$ to form the discrete right hand side.
The paper \cite{QuBianChen} does not define $\tilde v$, but I think I  know
what you did and it should be correct.  Numerical
testing should confirm that and demonstrate that the discretization is
correct.
I do it by implicitly adding $\tilde v$ to the discretization of $c$ as
part of the discretization of $c$.

I will test the Laplacian with a grid refinement study in 
the following way using $p=.8$ and $\delta=1$.
\begin{enumerate}
\item Evaluate $v$ on an $N \times N$ numerical grid with stepisze
$h=1/(N+1)$ on $\Lambda$.
\begin{enumerate}
\item Compute the analytic Laplacian with \eqnok{rlap} $L_e$.
\item Compute the finite difference Laplacian $L_d$. This computation
also tests the boundary data.
\end{enumerate}
\item Tabulate the relative error 
$E_r = \| L_e - L_d \|/\| L_e \|$ as a function 
of $N$ for the $\ell^1, \ell^2, \ell^\infty$ norms.
\end{enumerate}
We should see the errors decay by a factor of roughly four in all cases,
and they do.

\begin{table}[h]
\label{tab:laperr} 
\caption{Laplacian errors}
\centerline{
\begin{tabular}{llll} 
        N & $\ell^1$ & $\ell^2$ &$\ell^\infty$ \\ 
\hline 
10 & 3.894e-02 & 2.971e-02 & 2.400e-02   \\ 
20 & 9.974e-03 & 7.563e-03 & 5.808e-03   \\ 
40 & 2.534e-03 & 1.917e-03 & 1.431e-03   \\ 
80 & 6.394e-04 & 4.832e-04 & 3.552e-04   \\ 
\hline 
\end{tabular} :w
}
\end{table}

\section{The equations}
All seems to be fine in \cite{QuBianChen} except for the right hand side
function $c$. It should be
\begeq
\label{eq:cexact}
c = -\nabla^2 v + N(v) = 
(3/2) \sigma (\sigma + 1 ) T^{\sigma-1}
+ \frac{1}{r} (\sigma+1) T^\sigma + N(v).
\endeq
Recall that
\[
T = \frac{3 r - 1}{2} \max(0, r-1/3) \chi^*(r),
\sigma = 2p/(1-p)
\]
$\chi^*$ is the characteristic function of $(1/3,\infty)$, and so
\[
v = T^\sigma \max(0, r-1/3) = \frac{2}{3} T^{\sigma + 1}.
\]

In my discretization I do not multiply by $h^2$ as you do at the top of
page 19 in \cite{QuBianChen}. The reason for this is that I will want to look
at preconditioning and fast solvers expect the $h^{-2}$ to be part of the 
operator.

I will do another grid refinement study by computing the discrete form of
$H$ (which includes the boundary conditions) and evaluating the discrete form
of $H(v)$. This should be $O(h^2)$. This study confirms that the discretization
is consistent and that $v$ and $c$ are also consistent. 

\clearpage

\begin{table}[h]
\label{tab:hval}
\caption{Values of $\| H(v) \|/\| v \|$ as a function of $N$}
\centerline{
\begin{tabular}{llll} 
        n & $\ell^1$ & $\ell^2$ &$\ell^\infty$ \\ 
\hline 
10 & 4.705e+00 & 2.934e+00 & 2.082e+00   \\ 
20 & 1.127e+00 & 6.923e-01 & 4.454e-01   \\ 
40 & 2.772e-01 & 1.694e-01 & 1.030e-01   \\ 
80 & 6.884e-02 & 4.199e-02 & 2.476e-02   \\ 
\hline 
\end{tabular} 
}
\end{table}

\section{Solving with AA}

We will consider two formulations. The original approach from 
\cite{QuBianChen} is a differential equations formulation and
we begin with that one in \S~\ref{subsec:deform}.

The natural approach, which leads directly to the integral equations 
formulation in \S~\ref{subsec:inteq}, is to
use an initial iterate that satisfies the boundary conditions, then
the line search is not needed and one can use AA directly. In particular,
the intial iterate will be the solution of Laplace's equation
\begeq
\label{eq:laplace}
-\nabla^2 u_0 = 0 \mbox{ in $\Lambda$ } u_0 = v \mbox{ on $\partial \lambda$}.
\endeq

We can use \eqnok{laplace} to write the solution as in integral form.
We define $\cals(f;v)$ to be the solution operator of \eqnok{laplace}.
If $v = 0$ the  $\cals(f;0) = \call(f)$ is an linear integral operator
\[
\call f(x,y) = \int_\Lambda G(x,y,\xi, \eta) f(\xi,\eta) dA
\]
where 
\[
G(x,y,\xi, \eta) = \frac{1}{4 \pi} \ln((\xi-x)^2 + (\eta - y)^2).
\]
Then
\[
\cals(f;v) = \call(f) + \cals(0;v).
\]

Of course, when we compute $\call(f)$ we will use an efficient
solver rather than evaluate the two dimensional integral.

\subsection{Differential Equations Formulation}
\label{subsec:deform}
For the computions in this section
I report in this section I use (in your terminology)
\[
\gamma = .1
\]
and multiply the discrete residual by $h^2$ as you do. 

My discretiztion is (I think) exactly the same as yours. 

With this intial iterate AA does very well. Here are some results for
$p=.8$, $m=0, 1, 2, 4$, and $n = 900, 1600, 2500$.

\begin{figure}[h!]
\caption{\label{fig:deform} AA Performance: Original formulation}
\centerline{
\includegraphics[width=4.5in]{DE.pdf}
}
\end{figure}

It's interesting to see that the performance of
$AA(2)$ and $AA(4)$ does not degrade as rapidly with $n$
as $AA(0)$ and $AA(1)$. 

\subsection{Integral Equation Formulation}
\label{subsec:inteq}

\clearpage


\bibliographystyle{siamplain}
\bibliography{Comments}


\end{document}
